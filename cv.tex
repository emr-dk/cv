%% start of file `template.tex'.
%% Copyright 2006-2013 Xavier Danaux (xdanaux@gmail.com).
%
% This work may be distributed and/or modified under the
% conditions of the LaTeX Project Public License version 1.3c,
% available at http://www.latex-project.org/lppl/.


\documentclass[11pt,a4paper,roman]{moderncv}        % possible options include font size ('10pt', '11pt' and '12pt'), paper size ('a4paper', 'letterpaper', 'a5paper', 'legalpaper', 'executivepaper' and 'landscape') and font family ('sans' and 'roman')

% modern themes
\moderncvstyle{banking}                            % style options are 'casual' (default), 'classic', 'oldstyle' and 'banking'
\moderncvcolor{blue}                                % color options 'blue' (default), 'orange', 'green', 'red', 'purple', 'grey' and 'black'
%\renewcommand{\familydefault}{\sfdefault}         % to set the default font; use '\sfdefault' for the default sans serif font, '\rmdefault' for the default roman one, or any tex font name
%\nopagenumbers{}                                  % uncomment to suppress automatic page numbering for CVs longer than one page

% character encoding
\usepackage[utf8]{inputenc}                       % if you are not using xelatex ou lualatex, replace by the encoding you are using
%\usepackage{CJKutf8}                              % if you need to use CJK to typeset your resume in Chinese, Japanese or Korean
\usepackage{fontspec}
\defaultfontfeatures{Mapping=tex-text,Scale=MatchLowercase}
\setmainfont{Garamond Libre}
\setmonofont{Garamond Libre}
% adjust the page margins
%\usepackage[scale=0.8]{geometry}
 \usepackage[left=2cm, right=2cm, top=1.7cm]{geometry}
%\setlength{\hintscolumnwidth}{3cm}                % if you want to change the width of the column with the dates
%\setlength{\makecvheadnamewidth}{10cm}           % for the 'classic' style, if you want to force the width allocated to your name and avoid line breaks. be careful though, the length is normally calculated to avoid any overlap with your personal info; use this at your own typographical risks...

\usepackage{import}

% personal data
\name{Emil Møller}{Rasmussen} 

\address{Egilsgade 13, 5. tv. 2300 København S}% optional, remove / comment the line if not wanted; the "postcode city" and and "country" arguments can be omitted or provided empty
\phone[mobile]{+45 29451921}      % optional, remove / comment the line if not wanted
%\phone[fixed]{01234 123456}                    % optional, remove / comment the line if not wanted
%\phone[fax]{+3~(456)~789~012}                      % optional, remove / comment the line if not wanted
\email{emil.moeller.rasmussen@gmail.com}                               % optional, remove / comment the line if not wanted
%\homepage{()}                         % optional, remove / comment the line if not wanted
%\extrainfo{additional information}                 % optional, remove / comment the line if not wanted
%\photo[64pt][0.4pt]{picture}                       % optional, remove / comment the line if not wanted; '64pt' is the height the picture must be resized to, 0.4pt is the thickness of the frame around it (put it to 0pt for no frame) and 'picture' is the name of the picture file
%\quote{Some quote}                                 % optional, remove / comment the line if not wanted

% to show numerical labels in the bibliography (default is to show no labels); only useful if you make citations in your resume
%\makeatletter
%\renewcommand*{\bibliographyitemlabel}{\@biblabel{\arabic{enumiv}}}
%\makeatother
%\renewcommand*{\bibliographyitemlabel}{[\arabic{enumiv}]}% CONSIDER REPLACING THE ABOVE BY THIS

% bibliography with mutiple entries
%\usepackage{multibib}
%\newcites{book,misc}{{Books},{Others}}
%----------------------------------------------------------------------------------
%            content
%----------------------------------------------------------------------------------
\begin{document}
%\begin{CJK*}{UTF8}{gbsn}                          % to typeset your resume in Chinese using CJK
%-----       resume       ---------------------------------------------------------
\makecvtitle 
\vspace{-20pt}

\section{Resumé}
Jeg er uddannet indenfor Geografi og Geoinformatik og har igennem flere år arbejdet med data og analyser i en række forskellige kontekster. Jeg er særligt motiveret af at arbejde med data, når man løser problemstillinger, der kan gøre en forskel i den virkelige verden, hvorfor jeg i høj grad har arbejdet med klima- og miljøorienterede problemstillinger. Min erfaring spænder med alt fra at opsætte og integrere datainfrastruktur til at bearbejde store datamængder som fx batymetri og AIS-data samt udarbejde og visualisere analyser.
\\ \\
Som person er jeg positiv og har en løsnings- og resultatorienteret tilgang til mine arbejdsopgaver. Jeg er vant til at arbejde med mange mennesker fra forskelligartede faglige baggrunde og trives i dynamiske arbejdsmiljøer, hvor jeg har mulighed for at lære nyt. 

\section{Erfaring}

\vspace{4pt}

\begin{itemize}

\item{\cventry{Nuværende -- August 2017}{Udvikler og analytiker}{Bydata - Københavns Kommune}{}{}{\vspace{3pt} 
\begin{itemize}
\item Arkitekt bag total ombygning af eksisterende datainfrastruktur fra perl-baseret software til postgres og FME-baseret.
\item Udarbejdet analyse af $CO^{2}$ udledning fra skibe i Københavns Havn. Involverede indsamling og bearbejdning af AIS data for hele Danmark. Kombineret Python og Postgres opgave
\item Drivkraft bag implementering og udvikling af remote sensing i forvaltningen herunder brug af NDVI, UHI og besfæstelse.
\item Udviklet webapplikation til at downloade og bearbejde Sentinel-2 data.
\item Formidle og visualisere resultater fra analyser vha. Power BI, QGIS og lignende værktøjer.
\item Generelt arbejdet med at opsøge og understøtte datamæssige behov i forvaltningen, samt udarbejdelse af analyser. 
\end{itemize}
}}

\vspace{4pt}

\item{\cventry{Maj 2016 - August 2017}{GIS Udvikler}{Arktis og Forsvar - Styrelsen for Dataforsyning og Effektivisering}{}{}{\vspace{3pt}
\begin{itemize}
\item  Projektleder og rådgiver på SDI og webbaseret datadistributionsplatform for Forsvarets Materiel- og Indkøbsstyrelse.
\item  Produktion af dataprodukter og visualiseringer til brug for Forsvaret, samt andre relevante kontorer i styrelsen.
\item Udvikling af løsning til at generere grundkort til brug i håndholdte GPS'er baseret på open source løsninger.
\item Udvikling af automatisk proces til at skabe og opdatere verdensdækkende Sentinel-2 baggrundskort.
\item Arbejde med proces omkring generering af kort til redningshelikoptere og fly.
\item Generel udvikling af nye dataanvendelser, visualiseringer og kort, til brug i og udenfor kontoret og styrelsen.
\item Udforsk nye teknologiske muligheder for at løse kontorets opgaver, herunder afsøgning af databaser, filformater og værktøjer såsom ETL værktøjet FME.
\end{itemize}
}}

\vspace{4pt}

\item{\cventry{August 2015 - Maj 2016}{GIS-specialist}{Søkart Danmark og Forvaltning - Geodatastyrelsen}{}{}{\vspace{3pt}
\begin{itemize}
\item Arbejde med analyse og udvikling af dataprodukter med udgangspunkt i den maritime sektor, herunder bearbejdninger af bathymetrisk data. 
\item Afklaring og implementering af databasebaseret produktionssystem baseret på Teledyne CARIS GIS.
\item Generel database management og udvikling.
\item Udvikling af dataleverancer til relevante myndigheder og styrelser på ad-hoc behov.
\end{itemize}
}}

\vspace{4pt}

\item{\cventry{April 2012, Januar - Maj 2016, November 2017}{Selvstændig GIS Konsulent}{Diverse}{}{}{\vspace{3pt}
\begin{itemize}
\item Udarbejdelse af analyse over brugen af Twitter og sociale medier for Nordregio. Inkluderede hele workflow fra scraping til analyse til visualisering af trends indenfor det arktiske område.
\item Undervist i brugen af data- og analyseværktøjer for teknisk personale i Kujalleq Kommune.
\item  Udarbejd statistiske analyser, tabeller, figurer og kort til brug i akademisk kontekst.
\end{itemize}
}}

\vspace{4pt}

\item{\cventry{2012 - 2015}{Diverse studenterstillinger}{Diverse}{}{}{\vspace{3pt}
\begin{itemize}
\item  En række GIS-orienterede studenterstillinger, herunder: 
\begin{itemize} 
    \item Beredskab og Forsvar i Kort- og Matrikelstyrelsen.
    \item Hørsholm Kommune
    \item Naturstyrelsen
\end{itemize}
\end{itemize}
}}
\end{itemize}

\section{Uddannelse}

\vspace{4pt}

\begin{itemize}

\item{\cventry{2021 -- 2022}{København}{Enkelt fag i Økonomi}{Københavns Universitet}{\textit{Danmark}}{}}
\begin{itemize}
    \item Økonometri
    \item Anvendt Økonometri
\end{itemize}

\item{\cventry{2013 -- 2015}{København}{Master of Science i Geoinformatik}{Aalborg Universitet}{\textit{Danmark}}{}}

\item{\cventry{April 2014 -- Maj 2014}{Geneve}{Geoinformation in Disaster Situations}{UNOSAT / UNITAR / Københavns Universitet}{\textit{Schweiz}}{}}

\item{\cventry{2009 -- 2013}{København}{Geografi og Geoinformatik}{Københavns Universitet}{\textit{Danmark}}{}}
\begin{itemize}

\item{\cventry{2012}{Cape Town}{Miljøvidenskab}{University of Cape Town}{\textit{Sydafrika}}{}}
\end{itemize}
\end{itemize}

\vspace{2pt}

\section{Tekniske kundskaber}
Python (Pandas, geoPandas, Sci-kit, rasterIO, fiona, shapely), JavaScript (Node), FME, FME Server, ESRI ArcGIS, cloud computing, R (tidyverse), PostgreSQL + Postgis, MongoDB, Ubuntu / Linux, Windows, git, QGIS
\section{Organisatorisk og frivilligt arbejde}
\begin{itemize}
    \item{\cventry{Nuværende -- 2019}{København}{Kommunal repræsentant i Copernicus User Forum}{SDFI}{\textit{Danmark}}{}}
    \begin{itemize} 
    \item Én af to der repræsenterer kommunale interesser i det nationale CUF.
    \end{itemize} 
    
    \vspace{2pt}

    \item{\cventry{Nuværende -- 2014}{København}{Satellitdataudvalget, Kortdagsudvalget samt Arrangementsudvalg Øst }{Geoforum}{\textit{Danmark}}{}}
    \begin{itemize} 
    \item Har brugt flere år som studenterrepræsentant og frivillig i Geoforum.
    \item Frivillig i et udvalg der arbejder for at brede brugen af satellitdata blandt medlemmer.
    \end{itemize} 
    
\end{itemize}

\section{Sprog}
\vspace{1pt}
\begin{itemize}
\item \textbf{Dansk:} Modersmål 
\vspace{1pt}
\item \textbf{Engelsk:} Flydende 
\vspace{1pt}
\item \textbf{Fransk:} Godt kendskab
\vspace{1pt}
\item \textbf{Russisk:} Kendskab
\end{itemize}
\section{Personligt}

% Publications from a BibTeX file without multibib
%  for numerical labels: \renewcommand{\bibliographyitemlabel}{\@biblabel{\arabic{enumiv}}}% CONSIDER MERGING WITH PREAMBLE PART
%  to redefine the heading string ("Publications"): \renewcommand{\refname}{Articles}
\nocite{*}
\bibliographystyle{plain}
\bibliography{publications}                        % 'publications' is the name of a BibTeX file

% Publications from a BibTeX file using the multibib package
%\section{Publications}
%\nocitebook{book1,book2}
%\bibliographystylebook{plain}
%\bibliographybook{publications}                   % 'publications' is the name of a BibTeX file
%\nocitemisc{misc1,misc2,misc3}
%\bibliographystylemisc{plain}
%\bibliographymisc{publications}                   % 'publications' is the name of a BibTeX file

%-----       letter       ---------------------------------------------------------

\end{document}


%% end of file `template.tex'.
